\documentclass[14pt, openany]{book}
\pagestyle{plain}
\usepackage{setspace}
\usepackage{enumitem}
\usepackage[utf8]{inputenc}
\usepackage[T1]{fontenc}
\usepackage{extsizes}
\usepackage[english,russian]{babel}
\usepackage{geometry}
\usepackage{graphicx}
\usepackage{tempora}
\usepackage{amsmath}
\usepackage[nottoc,notlot,notlof,numbib]{tocbibind}
\usepackage{natbib}
\usepackage{indentfirst}
\usepackage{titlesec}
\titleformat{\chapter}[display]
  {\normalfont\bfseries}{}{0pt}{\Huge}
\usepackage{listings}
\usepackage{xcolor}
\usepackage{hyperref}
\hypersetup{
    colorlinks,
    citecolor=black,
    filecolor=black,
    linkcolor=black,
    urlcolor=black
}

\definecolor{codegreen}{rgb}{0,0.6,0}
\definecolor{codegray}{rgb}{0.5,0.5,0.5}
\definecolor{codepurple}{rgb}{0.58,0,0.82}
\definecolor{backcolour}{rgb}{0.98,0.98,0.98}

\lstdefinestyle{mystyle}{
    frame=single,
    aboveskip=5mm,
    belowskip=5mm,
    backgroundcolor=\color{backcolour},   
    language=scala,
    commentstyle=\color{codegreen},
    keywordstyle=\color{blue},
    numberstyle=\small\color{codegray},
    stringstyle=\color{codepurple},
    basicstyle=\ttfamily\footnotesize,
    breakatwhitespace=false,         
    breaklines=true,                 
    captionpos=b,                    
    keepspaces=true,                 
    numbers=left,                    
    numbersep=7pt,                  
    showspaces=false,                
    showstringspaces=false,
    showtabs=false,                  
    tabsize=2,
}

\lstset{style=mystyle}
  
\geometry{
    a4paper,
    left=20mm,
    top=20mm,
    right=20mm,
    bottom=20mm
}
\setlist[itemize]{noitemsep, topsep=0pt}
\setstretch{1.5}
\newcommand{\tla}{TLA\textsuperscript{+}}

\title{Применение формальных методов при спецификации бизнес-процессов}
\author{Морковкин Василий}
\date{2022}
\setlength{\parindent}{2em}

\begin{document}
\maketitle
\chapter*{Аннотация}
\par
История отрасли информационных технологий насчитывает немало случаев масштабных нарушений работы из-за изъянов программного обеспечения. Изъяны могут нести финансовые и репутационные потери, а также ставить под угрозу безопасность персональных данных и жизнь человека. Поэтому неотъемлемой частью процесса разработки программного обсепечения является поиск ошибок и их исправление.

Ошибки могут появляться на этапах:
\begin{itemize}
  \item спецификации,
  \item написания кода.
\end{itemize}

Практика покрытия кода автоматизированными тестами помогает минимизировать ошибки этапа написания кода, хорошо описана и широко применяется в индустрии. Ошибки спецификации, однако, такими тестами не обнаруживаются. Они могут проявляться в виде нарушения инваринтов работы системы и даже противоречивости постановки задачи. О возможных противоречиях в требованиях и теоретически достижимых гарантиях любой системы лучше знать еще до начала ее разработки. 

С этой целью данная работа фокусируется на разработке, эксплуатации и анализе применимости метода спецификации бизнес-процессов. За основу берутся формальные методы к верификации. Метод должен осваиваться разработчиками за разумное время, а результат его применения оправдывать расходы на использование.


\setcounter{page}{1}
\tableofcontents
\clearpage



\chapter{Введение}
В данной главе будут введены основные понятия, сформулированы цели, приведен анализ существующих подходов и описана структура работы.

\section{Основные понятия}
В контексте данной работы введем следующие понятия.

\emph{Формальные методы} --- набор математических техник, применяемых в сфере информационных технологий для формализации рассуждений и построения систем с изученными свойствами.

\emph{Спецификация} --- процесс разработки технического задания, которое может быть транслировано разработчиками в программный код.

\emph{Бизнес-процесс} --- упорядоченный набор действий, выполняемых людьми или машинами, результатом исполнения которых является продукт или услуга, потребляемые пользователями.

\section{Цели работы}

\begin{itemize}
  \item Исследование существующих подходов к валидации спецификаций бизнес-процессов.
  \item Разработка методики спецификации бизнес-процессов с применением формальных методов.
  \item Подготовка инструментария для использования методики в продуктовой разработке.
  \item Применение методики на практике, оценка затрат, преимуществ и ограничений.
\end{itemize}

\section{Анализ существующих подходов}

\section{Структура работы}

В Главе 2 рассматриваются математеческие формализмы, пригодные для формирования основы методики. Поясняется выбор языка спецификаций \tla. Описываются его базовые конструкции и выразительные возможности.

В Главе 3 разрабатывается сама методика. Приводятся детальные примеры использования и анализ опыта реального использования в продуктовой разработке. 

В Глава 4 описывается разработка инструментария \tla, полезного для применения методики.

Работа завершается подведением итогов и обсуждением дальнейших путей развития.

\chapter{Выбор инструмента}

\section{Моделирование времени}

\section{Сравнение технологий}

\section{Возможности \tla}

\chapter{Метод}

\section{Языки и нотации}

\section{Уточнение требований}

\section{Шаблоны}

\subsection{Алгоритм}

\subsection{Машина состояний}

\section{Примеры}

\subsection{Денежный перевод}

\subsection{Светофоры}

\section{Анализ применимости}

\subsection{Стоимость}

\subsection{Достоинства}

\subsection{Ограничения}

\chapter{Доработка инструментов \tla}

\section{Подсветка синтаксиса}

\section{Neovim плагин}

\chapter{Выводы}

\bibliographystyle{unsrt}
\bibliography{references}

\end{document}

